\chapter{Site Customisation}

{\em a discussion of how to customise each site.  Below is the existing content that will be restructured }


The Web Interface has been developed in PHP using the Zend Framework.  The Zend Framework implements a Model-View-Controller architecture.  
The following components of the Web Interface can be changed for Sahara installations.
\begin{itemize}
	\item Rig Type Page � this page is rig specific. If it is not written, a default page is displayed for that rig type.
	\item Customisable site header 
	\item Laboratory Rigs page 
	\item News page 
	\item FAQ page 
	\item Contact Us page 
	\item Configuration parameters (eg email addresses for feedback)
\end{itemize}

The Web Interface configuration file /config/config.ini has some configurable parameters for this. 

\section{Site Customisation}
\subsection{Header}
The header and footer are included on each page of the Sahara Site.  These can be customised using the configuration file or further if needed by changing the header partial script.  The configuration parameters that can be are:
\begin{itemize}
	\item header.logoGraphic � the institution graphic displayed on the top left 
	\item header.logoLink � the link to follow when clicking on the institution logo.
	\item header.nameImage � the image to be displayed to the right of the institution logo
\end{itemize}

The header images need to fit into the following dimensions to prevent cropping or resizing:
Height � 70px
Header Image width � 580px
	
\subsection{Feedback Form}
The feedback form is overlayed on whichever page is shown when the Feedback button is pressed.  The configurable parameters for the feedback are the email addresses to which the feedback should be sent eg

feedback.address[] = mdiponio@eng.uts.edu.au
feedback.address[] = tmachet@eng.uts.edu.au

\subsection{Laboratory Rig Page}
{\em update to new galleria}
There is a Laboratory Rig page which currently holds a gallery style display of images.  

If a similar gallery style display is to be implemented, this can be done by creating the institute specific script INSTALL\_DIR/institution/<Institution\_name>/Labinfo.php, where institution\_name matches that institution name configured in the config.ini file for the Web Interface.  The class name of the script must be <Institution>\_<Info\_type>.php (fin order for Zend�s auto loading to work).

This script should have a list of arrays with each array containing the information for the image.  The image information is an associative array containing:

filename => image filename (from baseUrl)

alt => subtitle for image

title => title for image

For example:
  	private \$Images = array (
    		array( "filename" => "uts/images/image1.jpg",
			"alt" => "Description of Image 1",
		"title" => "Title of image 1"),
   	 array( "filename" => "uts/images/image2.jpg",
		"alt" => "Description of Image 2",
		"title" => "Title of image 2"),
   	 array( "filename" => "uts/images/image3.jpg",
		"alt" => "Description of Image 3",
		"title" => "Title of image 3"));

If the gallery style is not wanted, this page can be replaced with whatever laboratory information is desired by overwriting the file  INSTALL\_DIR/application/views/scripts/labinfo/index.phtml.

\subsection{Information Pages}
The �information� pages cover the Contacts Us page, News page and Frequently Asked Questions (FAQ) page.

Similarly to the Laboratory Information page, the other information pages can be written for the specific institute or the default can be used, or completely overwritten.

For creating an institute specific Contacts Us page, create the script INSTALL\_DIR/institution/<Institution\_name>/Contacts.php, where institution\_name matches that institution name configured in the config.ini file for the Web Interface.  

This script should set up the information to be displayed on the Contact Us page in the table shown.  The structure is an associative array which contains role => contact information.  The contact information is again an associative array with any key => value pair that should be displayed (they will be displayed in a table as field => value).
The script should have a method getContacts() which returns this array. 

Any links must have the html mark up included.

For example:
private \$contacts = array (
"Operational" => array("Contact Name:" => "Michel de la Villefromoy",
				  "Contact Phone:" => "(02) 9514 2406",
				  "Contact Address:" => "UTS Building 1",
				  "" => "CB01.23.16",
			  "Contact Email:" => "<a href=\"mailto:name@address\">Operational</a>"),
	 "Technical" => array("Contact Name:" => "Tania Machet",
				  "Contact Phone:" => "(02) 9514 2975",
				  "Contact Address:" => "UTS Building 1",
				  "" => "CB01.23.16",
			  "Contact Email:" => "<a href=\"mailto:name@address\">Technical</a>"));

For creating an institute specific FAQ page, create the script INSTALL\_DIR/institution/<Institution\_name>/FAQ.php, where institution\_name matches that institution name configured in the config.ini file for the Web Interface.  
This script should set up the information to be displayed on the FAQ page in the default accordion style.  The structure is list of associative arrays which contain the key value pairs:
	question => question text
	answer =>  answer text
Any links or editing must have the html mark up included.
The script should have a method getFAQ() which returns this array.
For example:
     private \$FAQ = array( 
	 array("question" => "What is Sahara?",
"answer" => "Sahara is the software used to access remote laboratories anywhere, any time.  Sahara is a suite of open source software components that has been developed at UTS under the LabShare program. <p>  You are currently using Sahara V1.0. There will be further releases and	updates available from our repository at our <a href=\"http://sourceforge.net/projects/labshare-sahara/\"> 
open source repository</a>. <p> For more information or any questions regarding Sahara, please use the \"Send Feedback\" tab or the contact details provided."),
			
	array( "question" => "So what is a remote laboratory anyway?",
"answer" => "A remote lab is a set of laboratory apparatus and equipment which is configured	for remote usage over a network - usually the Internet. As much as possible, in setting up a remote laboratory, the goal should be to preserve the same apparatus and equipment (with the same limitations and imperfections) as would be used if the students were proximate to the equipment as per a conventional laboratory."));

For creating an institute specific News page, create the script INSTALL\_DIR/institution/<Institution\_name>/News.php, where institution\_name matches that institution name configured in the config.ini file for the Web Interface.  
This script should set up the information to be displayed on the News page in the default accordion style.  The structure is list of associative arrays which contain the key value pairs:
	header => News item header text
	info  =>  News item information
Any links or editing must have the html mark up included.
The script should have a method getNews() which returns this array.
